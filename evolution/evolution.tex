\documentclass{beamer}

\usepackage{polyglossia}
\setdefaultlanguage[spelling=new, babelshorthands=true]{german}

\usepackage{amssymb}
\usepackage{pifont}
\usepackage{graphicx}
\usepackage{minted}
\usepackage{epigraph}
\usepackage{default}

\newcommand{\cmark}{\ding{51}}
\newcommand{\xmark}{\ding{55}}

\title{mitrax}
\author{Benjamin Buch}
\date{18. Oktober 2016}

\subtitle{Eine moderne Bibliothek für Matritzen und Bildverarbeitung.}


\begin{document}
%\begin{frame}
%    \frametitle{Designentscheidungen für die mitrax-Bibliothek}
%\end{frame}
    \maketitle
\begin{frame}
    \frametitle{Kernforderungen an mitrax}
    \begin{itemize}
        \item Benutzerfreundlichkeit\\
        \begin{quote}»Make simple tasks simple!« \hspace{1em} — Bjarne Stroustrup\end{quote}
        \item Erweiterbarkeit\\
        \hspace{1em}
        \item Performance („zero overhead“)\\
        \begin{quote}»Don’t pay for what you don’t use.«\end{quote}
        \item Support für \mintinline{cpp}{constexpr}\\
        \hspace{1em}
    \end{itemize}
\end{frame}
\begin{frame}
    \frametitle{Was soll die »matrix« Schnittstelle bieten}
    \begin{itemize}
        \item Einheitliches Interface für unterschiedliche Implementierungen
        \item »Zero Overhead«
        \item Bereitstellung von Standardfunktionen (Implementierung muss nur wenige Basisfunktionen bereitstellen)
        \item Erweiterbarkeit
    \end{itemize}
\end{frame}
\begin{frame}
    \frametitle{Mögliche Ansätz für die Schnittstelle »matrix«}
    \begin{itemize}
        \item Als Basisklasse der Implementierungen (virtuelle Funktionen zum Zugriff)
        \item Vererbung mit abgeleiteter Klasse als Template-Argument
        \item Implementierung als Template-Argument ohne Vererbung
        \item Schnittstelle als Mixin für die Implementierungen
    \end{itemize}
\end{frame}

\end{document}
